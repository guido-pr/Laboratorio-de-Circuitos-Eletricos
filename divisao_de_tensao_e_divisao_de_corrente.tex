\documentclass[a4paper, 12pt]{article}

% --- PACOTES ---
\usepackage[utf8]{inputenc}
\usepackage[brazil]{babel}
\usepackage{lmodern}
\usepackage{geometry}
\geometry{a4paper, margin=2.5cm}
\usepackage{graphicx}
\usepackage{amsmath, amssymb}        % Pacote para ambientes matemáticos como 'align*' e 'cases'
\usepackage{float}          % Para melhor controle de posicionamento [H]
\usepackage{siunitx}        % Para formatar unidades (ex: 5 V, 1 kOhm)
\sisetup{
output-decimal-marker = {,}, % Usa vírgula como separador decimal
per-mode = symbol
}

% --- INFORMAÇÕES DO DOCUMENTO ---
\title{Laboratório de Circuitos Elétricos \\ Divisão de tensão e divisão de corrente}
\author{Gabriel Franklin Lopes de Oliveira \\
Guido Piancastelli Ramos \\
Matheus Muniz Soares
}
\date{3 de Setembro de 2025}

\begin{document}

\maketitle

%======================================================================
% SEÇÃO 1: INTRODUÇÃO
%======================================================================
\section{Introdução}

Na análise de circuitos elétricos, a capacidade de simplificar configurações complexas é essencial. Embora métodos robustos como as Leis de Kirchhoff permitam resolver qualquer circuito linear, o reconhecimento de padrões recorrentes oferece "atalhos" analíticos que aceleram o projeto e a compreensão intuitiva do comportamento do circuito. Entre os mais importantes desses padrões estão os circuitos divisores de tensão e de corrente, e a sua engenhosa combinação na Ponte de Wheatstone.

Estes circuitos não são apenas artifícios teóricos, mas sim blocos de construção fundamentais em praticamente todos os domínios da eletrônica, desde a polarização de transistores até o funcionamento de sensores de alta precisão.

\subsection{O Princípio do Divisor de Tensão}
O divisor de tensão é uma configuração que consiste em dois ou mais resistores conectados em série a uma fonte de tensão. Sua principal função é produzir uma tensão de saída ($V_{out}$) que seja uma fração conhecida da tensão de entrada ($V_s$). Para um circuito com dois resistores, $R_1$ e $R_2$, a corrente total $I$ que os atravessa é dada pela Lei de Ohm aplicada à resistência série total:
\[
I = \frac{V_s}{R_1 + R_2}
\]
A tensão sobre o resistor $R_2$, que corresponde à saída $V_{out}$, é simplesmente $V_{out} = I \cdot R_2$. Substituindo a expressão da corrente, obtemos a fórmula do \textbf{divisor de tensão}:
\begin{equation}
V_{out} = V_s \cdot \frac{R_2}{R_1 + R_2}
\label{eq:divisor_tensao}
\end{equation}
A equação demonstra que a tensão de entrada é dividida entre os resistores de forma diretamente proporcional aos seus valores de resistência.

\subsection{O Princípio do Divisor de Corrente}
De forma análoga, o divisor de corrente é o conceito dual que descreve como uma corrente de entrada ($I_s$) se distribui entre dois ou mais ramos paralelos. Em uma configuração com dois resistores em paralelo, $R_1$ e $R_2$, a tensão $V$ é a mesma em ambos. A corrente total se divide de acordo com a Lei das Correntes de Kirchhoff: $I_s = I_1 + I_2$.

Para encontrar a corrente $I_2$ que flui através de $R_2$, podemos usar o fato de que a tensão no nó é $V = I_s \cdot (R_1 \parallel R_2)$. Assim, $I_2 = V/R_2$. Após a substituição e simplificação, chegamos à fórmula do \textbf{divisor de corrente}:
\begin{equation}
I_2 = I_s \cdot \frac{R_1}{R_1 + R_2}
\label{eq:divisor_corrente}
\end{equation}
Note que, ao contrário do divisor de tensão, a corrente que flui por um ramo é inversamente proporcional à sua resistência. Ou seja, a maior parcela da corrente sempre seguirá o caminho de menor resistência.

\subsection{A Ponte de Wheatstone}
A Ponte de Wheatstone representa uma das aplicações mais notáveis do princípio do divisor de tensão. O circuito consiste em dois divisores de tensão conectados em paralelo a uma mesma fonte $V_s$. Um medidor de tensão (galvanômetro) é conectado entre os pontos centrais dos dois divisores para medir a diferença de potencial, $V_d$.

A principal utilidade da ponte é a medição de uma resistência desconhecida ($R_x$) com altíssima precisão. Isso é alcançado quando a ponte está em \textbf{equilíbrio}, condição na qual a tensão diferencial $V_d$ é exatamente zero. Para que $V_d = 0$, as tensões nos pontos centrais de cada divisor devem ser idênticas. Igualando as equações do divisor de tensão para cada lado (assumindo uma configuração com resistores $R_1, R_2, R_3, R_x$), temos:
\[
V_s \cdot \frac{R_2}{R_1 + R_2} = V_s \cdot \frac{R_x}{R_3 + R_x}
\]
A simplificação algébrica desta igualdade leva à célebre \textbf{condição de equilíbrio da ponte}:
\begin{equation}
R_1 R_x = R_2 R_3
\label{eq:ponte_equilibrio}
\end{equation}
Esta relação permite calcular $R_x$ com base nos valores conhecidos dos outros três resistores. O princípio da ponte desequilibrada ($V_d \neq 0$) também é a base para inúmeros sensores que medem grandezas físicas como pressão e deformação.

O presente relatório tem como objetivo a verificação experimental destes três circuitos fundamentais. Serão comparados os valores teóricos, obtidos a partir das equações apresentadas, com os resultados de simulações computacionais e medições práticas, a fim de validar a teoria e quantificar as possíveis discrepâncias inerentes a um ambiente de laboratório.

%======================================================================
% SEÇÃO 2: OBJETIVOS
%======================================================================
\section{Objetivos}

Este experimento tem como finalidade aprofundar a compreensão teórica e prática da análise de circuitos em corrente contínua. Os objetivos são:

\begin{itemize}
\item \textbf{Geral:}  Verificar as propriedades dos circuitos básicos para atenuação de corrente e de
tensão. E verificar as propriedades da ponte de Wheatstone
\item \textbf{Específicos:}
\begin{itemize}
    \item Determinar teoricamente as correntes e as quedas de tensão nos componentes do circuito utilizando divisão de corrente e de tensão e ponte de Wheatstone.
    \item Realizar a montagem prática do circuito em protoboard.
    \item Medir experimentalmente as tensões e correntes no circuito utilizando um multímetro.
    \item Comparar os resultados obtidos através dos cálculos teóricos, da simulação computacional (LTspice) e das medições experimentais, analisando as possíveis fontes de erro.
\end{itemize}
\end{itemize}

%======================================================================
% SEÇÃO 3: CÁLCULOS TEÓRICOS
%======================================================================
\section{Cálculos teóricos}
\subsection{Divisor de Corrente}
\subsubsection{Forma genérica}

\begin{figure}[H]
\centering
\includegraphics[width=0.4\textwidth]{divisor_de_corrente.png}
\caption{Divisor de Corrente}
\label{fig:divisor_de_corrente}
\end{figure}

% ERRO CORRIGIDO: Removido comando \maketitle duplicado que estava aqui.

Utilizando o circuito da Figura~\ref{fig:divisor_de_corrente}, iremos determinar $I_1$ e $I_2$ em função de $I_s$, $R_1$ e $R_2$. Considere o nó superior comum a $R_1$ e $R_2$. Pela LKC (Lei de Kirchhoff das Correntes):
\begin{equation}
I_s = I_1 + I_2.
\end{equation}
A tensão no nó (em relação ao terra) é a mesma nos dois ramos:
\begin{equation}
V = I_1 R_1 = I_2 R_2.
\end{equation}
Logo,
\begin{equation}
I_s = \frac{V}{R_1} + \frac{V}{R_2}
= V\!\left(\frac{1}{R_1} + \frac{1}{R_2}\right)
\quad\Rightarrow\quad
V = I_s\,\frac{R_1 R_2}{R_1 + R_2}.
\end{equation}

Assim, as correntes são dadas pelo \emph{divisor de corrente}:
\begin{align}
I_1 &= \frac{V}{R_1} = I_s\,\frac{R_2}{R_1 + R_2},\\[6pt]
I_2 &= \frac{V}{R_2} = I_s\,\frac{R_1}{R_1 + R_2}.
\end{align}

\subsubsection{Substituindo os valores}
Agora, assumindo $R_{s} = \SI{1}{\kilo\ohm}$, $R_{1} = \SI{2.2}{\kilo\ohm}$ e $R_{2} = \SI{5.6}{\kilo\ohm}$:\\[3pt]

\begin{align}
R_1 \parallel R_2 &= \frac{2200 \times 5600}{2200 + 5600} 
= \frac{12320000}{7800} \approx \SI{1579,49}{\ohm}, \\[6pt] 
I_s &= \frac{10}{1000 + 1579,49} 
= \frac{10}{2579,49} 
\approx \SI{3,876}{\milli\ampere}.
\end{align}

\begin{align*}
I_1 &= \SI{3,876}{\milli\ampere} \cdot \frac{5600}{2200 + 5600} 
= \SI{3,876}{\milli\ampere} \cdot \frac{5600}{7800} 
\approx \SI{2,782}{\milli\ampere}, \\[6pt]
I_2 &= \SI{3,876}{\milli\ampere} \cdot \frac{2200}{7800} 
= \SI{3,876}{\milli\ampere} \cdot 0,2821 
\approx \SI{1,094}{\milli\ampere}.
\end{align*}

\subsection{Divisor de Tensão}
\subsubsection{Forma genérica}

\begin{figure}[H]
\centering
\includegraphics[width=0.4\textwidth]{divisor_de_tensao.png}
\caption{Divisor de Tensão}
\label{fig:divisor_de_tensao}
\end{figure}

Utilizando o circuito da Figura~\ref{fig:divisor_de_tensao}, iremos determinar $V_1$ e $V_2$ em função de $V_s$, $R_1$ e $R_2$.
A corrente é dada por:
\begin{equation}
I = \frac{V_s}{R_1 + R_2}.
\end{equation}
Logo, podemos escrever:
\begin{align}
V_1 &= I \cdot R_1 = V_s \cdot \frac{R_1}{R_1 + R_2}, \\[6pt]
V_2 &= I \cdot R_2 = V_s \cdot \frac{R_2}{R_1 + R_2}.
\end{align}

\subsubsection{Substituindo os valores}
Considerando $R_1 = \SI{5.6}{\kilo\ohm}$, $R_2 = \SI{1.2}{\kilo\ohm}$ e $V_s = \SI{5}{\volt}$.

\begin{align*}
V_1 &= 5 \cdot \frac{5600}{5600 + 1200} 
= 5 \cdot \frac{5600}{6800} 
\approx \SI{4,12}{\volt}, \\[6pt]
V_2 &= 5 \cdot \frac{1200}{5600 + 1200} 
= 5 \cdot \frac{1200}{6800} 
\approx \SI{0,88}{\volt}.
\end{align*}

Considerando $R_1 = \SI{5.6}{\kilo\ohm}$, $R_2 = \SI{5.6}{\kilo\ohm}$ e $V_s = \SI{5}{\volt}$.

\begin{align*}
V_1 &= 5 \cdot \frac{5600}{5600 + 5600} 
= 5 \cdot \frac{5600}{11200} 
= \SI{2,5}{\volt}, \\[6pt]
V_2 &= 5 \cdot \frac{5600}{5600 + 5600} 
= 5 \cdot \frac{5600}{11200} 
= \SI{2,5}{\volt}.
\end{align*}

Considerando $R_1 = \SI{8.2}{\mega\ohm}$, $R_2 = \SI{8.2}{\mega\ohm}$ e $V_s = \SI{5}{\volt}$.

\begin{align*}
V_1 &= 5 \cdot \frac{8200000}{8200000 + 8200000} 
= 5 \cdot \frac{8200000}{16400000} 
= \SI{2,5}{\volt}, \\[6pt]
V_2 &= 5 \cdot \frac{8200000}{8200000 + 8200000} 
= 5 \cdot \frac{8200000}{16400000} 
= \SI{2,5}{\volt}.
\end{align*}

\subsection{Ponte de Wheatstone}
\subsubsection{Forma genérica}

\begin{figure}[H]
\centering
\includegraphics[width=0.4\textwidth]{ponte_de_whetstone.png}
\caption{Ponte de Wheatstone}
\label{fig:ponte_de_wheatstone}
\end{figure}

Considere o circuito da Ponte de Wheatstone, representado na Figura~\ref{fig:ponte_de_wheatstone}. Iremos escrever a tensão $V_d$ em função de $V_s$, $R_1$ e $R_2$.
\\
O ponto central da esquerda tem tensão:
\[
V_{1\text{ esq}} = V_s \cdot \frac{R_1}{R_1 + R_{2\text{ esq}}}
\]
e, o ponto central da direita:
\[
V_{1\text{ dir}} = V_s \cdot \frac{R_1}{R_1 + R_{2\text{ dir}}}
\]
A tensão diferencial $V_d$ é
\[
V_d = V_{1\text{ esq}} - V_{1\text{ dir}}.
\]

\subsubsection{Substituindo os valores}
\subsubsection*{Caso 1: $R_1 = \SI{5.6}{\kilo\ohm}$, $R_{2\text{ esq}} = R_{2\text{dir}} = \SI{1.2}{\kilo\ohm}$}
\begin{align*}
V_{1\text{ esq}} &= 5 \cdot \frac{5600}{5600+1200} 
= 5 \cdot \frac{5600}{6800} 
\approx \SI{4,12}{\volt}, \\[6pt]
V_{1\text{dir}} &= V_{1\text{ esq}} \approx \SI{4,12}{\volt}, \\[6pt]
V_d &= 4,12 - 4,12 = \SI{0}{\volt}.
\end{align*}

\subsubsection*{Caso 2: $R_1 = \SI{5.6}{\kilo\ohm}$, $R_{2\text{ esq}} = \SI{1.2}{\kilo\ohm}$, $R_{2\text{ dir}} = \SI{1.0}{\kilo\ohm}$}
\begin{align*}
V_{1\text{ esq}} &= 5 \cdot \frac{5600}{5600+1200} 
= 5 \cdot \frac{5600}{6800} 
\approx \SI{4,12}{\volt}, \\[6pt]
V_{1\text{dir}} &= 5 \cdot \frac{5600}{5600+1000} 
= 5 \cdot \frac{5600}{6600} 
\approx \SI{4,24}{\volt}, \\[6pt]
V_d &= 4,12 - 4,24 \approx \SI{-0,12}{\volt}.
\end{align*}

%======================================================================
% SEÇÃO 4: PROCEDIMENTOS EXPERIMENTAIS
%======================================================================
\section{Procedimentos Experimentais}
A parte prática do experimento consistiu na montagem e medição de três configurações de circuitos elétricos para validar os conceitos teóricos de divisores de tensão, divisores de corrente e da ponte de Wheatstone. Os passos seguidos estão descritos abaixo.

\subsection{Materiais Utilizados}
\begin{itemize}
\item 1 Fonte de alimentação DC ajustável.
\item 1 Multímetro digital (utilizado como voltímetro, amperímetro e ohmímetro).
\item 1 Protoboard (matriz de contatos).
\item Resistores com os seguintes valores nominais: \SI{5,6}{\kilo\ohm}, \SI{1,2}{\kilo\ohm}, \SI{1}{\kilo\ohm} e \SI{8,2}{\mega\ohm}.
\item Fios de conexão (jumpers).
\end{itemize}

\subsection{Montagem e Medições}
Para cada circuito, a fonte de alimentação foi previamente ajustada e aferida com o multímetro para garantir a tensão de entrada correta.

\subsubsection{Divisor de Corrente}
\begin{enumerate}
\item \textbf{Montagem do Circuito:} O circuito divisor de corrente foi montado na protoboard conforme o diagrama da Figura 1.

\item \textbf{Medição das Correntes:} Com o circuito energizado, o multímetro foi configurado como amperímetro (DCA). As correntes nos ramos dos resistores R1 e R2 foram medidas inserindo-se o instrumento em série em cada ramo. Os valores medidos foram registrados na Tabela 6 para comparação com os valores calculados.
\end{enumerate}

\subsubsection{Divisor de Tensão}
\begin{enumerate}
\setcounter{enumi}{2} % Continua a numeração do passo anterior
\item \textbf{Montagem do Circuito:} O circuito divisor de tensão foi montado na protoboard, de acordo com o esquemático da Figura 2.

\item \textbf{Medição das Tensões:} Com o multímetro na função de voltímetro (DCV), foram medidas as quedas de tensão nos terminais dos resistores R1 e R2. As pontas de prova foram posicionadas em paralelo com cada componente. Os resultados foram anotados nas Tabelas 2 a 4.

\item \textbf{Modificação do Circuito (Resistores Iguais):} Os resistores foram substituídos por dois componentes de valor nominal igual a \SI{5,6}{\kilo\ohm}. As medições de tensão foram refeitas e os resultados discutidos.

\item \textbf{Análise com Alta Resistência:} O passo anterior foi repetido, desta vez utilizando resistores de \SI{8,2}{\mega\ohm}. Eventuais discrepâncias entre os valores medidos e esperados foram registradas para análise na seção de conclusão.
\end{enumerate}

\subsubsection{Ponte de Wheatstone}
\begin{enumerate}
\setcounter{enumi}{6} % Continua a numeração
\item \textbf{Montagem da Ponte:} O circuito da ponte de Wheatstone foi montado conforme a Figura 3, utilizando inicialmente R1 = \SI{5,6}{\kilo\ohm} e R2 = \SI{1,2}{\kilo\ohm}.

\item \textbf{Medição da Tensão de Desequilíbrio:} A tensão $V_d$ entre os nós centrais da ponte foi medida com o voltímetro. O valor foi registrado na Tabela 5.

\item \textbf{Alteração da Ponte:} O resistor R2 do lado direito da ponte foi substituído por um de \SI{1}{\kilo\ohm}. A medição da tensão $V_d$ foi realizada novamente para observar a mudança no desequilíbrio da ponte.
\end{enumerate}

%======================================================================
% SEÇÃO 5: SIMULAÇÃO E RESULTADOS
%======================================================================
\section{Simulação e Resultados}
Para validar os valores teóricos calculados e realizar as devidas comparações, o circuito foi simulado no software \textbf{LTspice}. Os resultados obtidos na simulação, juntamente com os valores calculados e os medidos experimentalmente, são apresentados nas tabelas a seguir, com as respectivas imagens do circuito na simulação:

\begin{figure}[H]
\centering
\includegraphics[width=0.4\textwidth]{sim_div_corrente.png}
\caption{Divisor de corrente}
\label{fig:sim_div_corrente}
\end{figure}

\begin{table}[H]
\centering
\caption{Valores de corrente}
\label{tab:correntes_div}
\begin{tabular}{|l|c|c|c|c|}
\hline
\textbf{Componente} & \textbf{Calculado (\si{\milli\ampere})} & \textbf{Simulado (\si{\milli\ampere})} & \textbf{Medido (\si{\milli\ampere})} & \textbf{E (\%)} \\
\hline
$I_s$ & 3,876 & 3,877 & 3,93 & 1,4 \\ \hline
$I_1$ & 2,782 & 2,783 & 2,82 & 1,4 \\ \hline
$I_2$ & 1,094 & 1,093 & 1,10 & 0,5 \\ \hline
\end{tabular}
\end{table}

\begin{figure}[H]
\centering
\includegraphics[width=0.4\textwidth]{sim_div_tensao1.png}
\caption{Divisor de tensão $R_1 = \SI{5,6}{\kilo\ohm}$ e $R_2 = \SI{1,2}{\kilo\ohm}$}
\label{fig:sim_div_tensao1}
\end{figure}

\begin{table}[H]
\centering
\caption{Valores de tensão ($R_1 = \SI{5,6}{\kilo\ohm}$ e $R_2 = \SI{1,2}{\kilo\ohm}$)}
\label{tab:tensoes_div1}
\begin{tabular}{|l|c|c|c|c|}
\hline
\textbf{Componente} & \textbf{Calculado (\si{\volt})} & \textbf{Simulado (\si{\volt})} & \textbf{Medido (\si{\volt})} & \textbf{E (\%)} \\
\hline
$V_1$ & 4,117 & 4,118 & 4,12 & 0,05 \\ \hline
$V_2$ & 0,882 & 0,882 & 0,87 & 1,36 \\ \hline
\end{tabular}
\end{table}

\begin{figure}[H]
\centering
\includegraphics[width=0.4\textwidth]{sim_div_tensao1.png}
\caption{Divisor de tensão $R_1 = R_2 = \SI{5,6}{\kilo\ohm}$}
\label{fig:sim_div_tensao1}
\end{figure}

\begin{table}[H]
\centering
\caption{Valores de tensão ($R_1 = R_2 = \SI{5,6}{\kilo\ohm}$)}
\label{tab:tensoes_div2}
\begin{tabular}{|l|c|c|c|c|}
\hline
\textbf{Componente} & \textbf{Calculado (\si{\volt})} & \textbf{Simulado (\si{\volt})} & \textbf{Medido (\si{\volt})} & \textbf{E (\%)} \\
\hline
$V_1$ & 2,5 & 2,5 & 2,52 & 0.80\\ \hline
$V_2$ & 2,5 & 2,5 & 2,47 & 1.20 \\ \hline
\end{tabular}
\end{table}

\begin{figure}[H]
\centering
\includegraphics[width=0.4\textwidth]{sim_div_tensao1.png}
\caption{Divisor de tensão $R_1 = R_2 = \SI{8,2}{\mega\ohm}$}
\label{fig:sim_div_tensao1}
\end{figure}

\begin{table}[H]
\centering
\caption{Valores de tensão ($R_1 = R_2 = \SI{8,2}{\mega\ohm}$)}
\label{tab:tensoes_div3}
\begin{tabular}{|l|c|c|c|c|}
\hline
\textbf{Componente} & \textbf{Calculado (\si{\volt})} & \textbf{Simulado (\si{\volt})} & \textbf{Medido (\si{\volt})} & \textbf{E (\%)} \\
\hline
$V_1$ & 2,5 & 2,5 & 1,76 & 29.2\\ \hline
$V_2$ & 2,5 & 2,5 & 1,77 & 29.6 \\ \hline
\end{tabular}
\end{table}

\begin{figure}[H]
\centering
\includegraphics[width=0.4\textwidth]{sim_ponte1.png}
\caption{Ponte de Wheatstone $V_{d1}$ para $R_2=\SI{1,2}{\kilo\ohm}$ e $V_{d2}$ para $R_2=\SI{1,2}{\kilo\ohm}$}
\label{fig:sim_ponte1}
\end{figure}

\begin{figure}[H]
\centering
\includegraphics[width=0.4\textwidth]{sim_ponte2.png}
\caption{Ponte de Wheatstone $V_{d1}$ para $R_2=\SI{1,2}{\kilo\ohm}$ e $V_{d2}$ para $R_2=\SI{1}{\kilo\ohm}$}
\label{fig:sim_ponte2}
\end{figure}

\begin{table}[H]
\centering
\caption{Valores de tensão da ponte ($V_{d1}$ para $R_2=\SI{1,2}{\kilo\ohm}$ e $V_{d2}$ para $R_2=\SI{1}{\kilo\ohm}$)}
\label{tab:tensoes_ponte} 
\begin{tabular}{|l|c|c|c|c|}
\hline
\textbf{Componente} & \textbf{Calculado (\si{\volt})} & \textbf{Simulado (\si{\volt})} & \textbf{Medido (\si{\volt})} & \textbf{E (\%)} \\
\hline
$V_{d1}$ & 0 & 0 & 0,015 & 0 \\ \hline
$V_{d2}$ & -0,12 & 0,12 & 0,118 & 0,02 \\ \hline
\end{tabular}
\end{table}



\begin{table}[H]
\centering
\caption{Valores de resistência}
\label{tab:resistencias}
\begin{tabular}{|l|c|c|c|}
\hline
\textbf{Componente} & \textbf{Nominal (\si{\kilo\ohm})} & \textbf{Medido (\si{\kilo\ohm})} & \textbf{E (\%)} \\
\hline
$R_0$ & 1,0 & 0,983 & 1,7 \\ \hline
$R_1$ & 1,2 & 1,192 & 0,7 \\ \hline
$R_2$ & 1,2 & 1,191 & 0,7 \\ \hline
$R_3$ & 2,2 & 2,15 & 2,3 \\ \hline
$R_4$ & 5,6 & 5,51 & 1,6 \\ \hline
$R_5$ & 5,6 & 5,62 & 0,4 \\ \hline
$R_6$ & 8200 & 8280 & 1,0 \\ \hline
$R_7$ & 8200 & 8260 & 0,7 \\ \hline
\end{tabular}
\end{table}

%======================================================================
% SEÇÃO 6: QUESTÕES PARA O RELATÓRIO
%======================================================================
\section{Questões para o Relatório}
As propriedades de divisão de corrente poderiam ser aplicadas para as correntes $I_{1}$ e $I_{2}$ no
circuito mostrado na Figura \ref{fig:questao_para_relatorio}? Explique.

\begin{figure}[H]
\centering
\includegraphics[width=0.4\textwidth]{questao_para_relatorio.png}
\caption{Questões para o Relatório}
\label{fig:questao_para_relatorio}
\end{figure}

De forma a obtermos $I_1$ e $I_2$ diretamente, não é possível, pois temos também a configuração dos resistores $R_3$, $R_4$ e $R_5$. Porém, consguimos utilizar resistência equivalente para encontrarmos um circuito com apenas dois resistores em paralelo ($R_1*$ = $(R_1+(R_3||(R_4+R_5)$ e $R_2*$ = $R_2$), onde obteremos $I_2$, e, após isso, encontrarmos $I_1$. 

%======================================================================
% SEÇÃO 7: CONCLUSÃO
%======================================================================
\section{Conclusão}
O presente trabalho experimental permitiu a verificação prática dos princípios fundamentais do divisor de tensão, do divisor de corrente e da Ponte de Wheatstone. Ao comparar os valores obtidos por meio de cálculos teóricos, simulação computacional no LTspice e medições em laboratório, foi possível validar a teoria com um alto grau de precisão na maioria dos casos, além de identificar as limitações inerentes aos instrumentos de medição.

A análise dos resultados demonstra que, para os circuitos com resistores na faixa de quilohms (\si{\kilo\ohm}), os valores medidos de tensão e corrente apresentaram excelente concordância com os valores teóricos e simulados. Os erros percentuais, em sua maioria inferiores a 2\%, são consistentes com as fontes de incerteza esperadas em um ambiente de laboratório, como a tolerância dos resistores (verificada na Tabela \ref{tab:resistencias}), pequenas flutuações na tensão da fonte e a precisão finita do multímetro.

Uma notável e instrutiva exceção foi observada no teste do divisor de tensão com resistores de \SI{8,2}{\mega\ohm}. Neste caso, a discrepância entre o valor teórico (\SI{2,5}{\volt}) e o medido (aproximadamente \SI{1,76}{\volt}) foi de quase 30\%. Este erro significativo não se deve a uma falha na teoria, mas sim ao \textbf{efeito de carga} (loading effect) do voltímetro. O multímetro digital possui uma resistência interna finita (tipicamente da ordem de \SI{10}{\mega\ohm}). Ao medir a tensão em um resistor de valor comparável (\SI{8,2}{\mega\ohm}), a resistência do próprio instrumento se torna um caminho alternativo para a corrente, alterando a resistência equivalente do ramo e, consequentemente, a tensão que se deseja medir. Este resultado prático evidencia de forma clara que a impedância de entrada do instrumento de medição é um fator crítico que não pode ser ignorado ao se trabalhar com circuitos de alta impedância.

Na análise da Ponte de Wheatstone, os resultados também foram conclusivos. A condição de equilíbrio foi quase perfeitamente atingida, com uma tensão diferencial medida de apenas \SI{15}{\milli\volt}, um valor residual atribuível às tolerâncias dos resistores. No caso desequilibrado, a tensão medida (\SI{0,118}{\volt}) foi extremamente próxima do valor calculado (\SI{0,12}{\volt}), validando o modelo matemático para a ponte.

Portanto, conclui-se que os objetivos do experimento foram plenamente alcançados. As leis de divisão de tensão e corrente foram empiricamente confirmadas, e a aplicação prática desses conceitos na Ponte de Wheatstone foi demonstrada com sucesso. Mais importante, o experimento proporcionou uma valiosa lição sobre a interação entre o circuito e o instrumento de medição, reforçando a importância de se considerar as características não ideais dos componentes e equipamentos em qualquer análise de engenharia.

%======================================================================
% SEÇÃO 8: REFERÊNCIAS
%======================================================================
\begin{thebibliography}{9}

\bibitem{nilsson}
NILSSON, James W.; RIEDEL, Susan A. \textit{Circuitos Elétricos}. 10. ed.

\end{thebibliography}

\end{document}
